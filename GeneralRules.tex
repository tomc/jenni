\documentclass[tom-ari]{subfile}
\begin{document}


	
\chapter{General Rules}

\rem{Note that these were copied verbatim from the Tom/Ari notes. I'm going to try to go through it and look for discrepancies but I may miss some.}

Some rules in no particular order.

\begin{itemize}[]
\item \textbf{Relays Off} Most relay auctions bail in competition, we don't tend to try to continue to relay when the opponents are interfering. The only exception is double, which is generally ignored. (Not true for \cl1 opener responses.)
\item \textbf{Late Doubles} In most extended auctions doubles are ignored and our responses remain the same. This is notably true over RKC style bids; there is no R0P1 or the like. Initial doubles by the opps can and often does change our response structure.
\item \textbf{Fast arrival} When we are forced to a certain level (say after a cuebid), bidding that level is the weakest action. This is true even over something like a double, where pass is more encouraging than retreating.
\item \textbf{Jumps} Most single jump shifts in comp are fit showing. Most jump raises are weak. Double jumps where available are splinters, but are lower precedence than Fit when only 1 jump available. Double jump cuebid is mixed when available.
\item \textbf{Cuebids} Our direct cuebids in response to partner's call generally show support, although it is possible that some one off cases may exist where you need to force with no good options.  Delayed cuebids are probing / generally ask for stoppers. When 2 opponent suits are available, we cuebid what we have not what we are looking for.
\item \textbf{2x Cuebid} As a psyche protection in \cl1 auctions, if we cuebid the opps suit twice that is \textit{natural}. This has come up in play and has proven useful.
\end{itemize}

\section{Forcing Passes}

Some special notes about forcing passes, as we on occasion need to deal with high level competition in forcing situations where we have done little to no description of our shape.

In general, if we bid directly in a FP situation that is more encouraging than if we pass and pull the double.  This is the opposite of ``standard'' - so called Inverted Pass and Pull.  Simple example:  \he1--(\sp2)--\sp3--(\sp4), \he5 would be forward going and pass and pulling a double of \sp4 would be to get out in \he5.

There can on occasion be situations where opener's hand is more or less undefined, and need to sort out what's going on at a high level.  Here's an example from play (bidding only, cards aren't relevant) from the JLall with what Tom thinks the bids should mean:

\begin{bidtable}{\compauct{1c,3s,4x,4s}}
	4NT & Encouraging in partner's suit, better than 5x. If x=Maj, this is RKC. \\
	5x & Mild encouragement \\
	5y & New suits are strong and natural but NF \\
	Dbl & Suggests defense \\
	Pass & Generally expects a double, then: \\
	& \rightarrow 4NT = 2 suited; can include a partial fit for partner as 1 of the suits \\
	& \rightarrow 5x = To play, no slam interest \\
	& \rightarrow 5y = My own suit, suggesting a contract, no slam interest \\
\end{bidtable}

At the table Tom bid 4NT which was interpreted (I think?) as 2 suited and we got overboard.  I think the general approach here is playable and also consistent with the philosophy espoused above.  It also matches the style elsewhere.  (Such as a free 2NT being good/bad, but a 2NT response over a double is scrambling.  Same idea here.)

\section{General defenses}

\begin{itemize}[]
\item \textbf{2 Suiter, known} We play lower cuebid for our lower suit, higher for higher. Double is a good hand, with a second double being penalty.

In general the cuebid is the stronger action; the only exception is when the high cuebid is below our low suit, then the bidding the high suit is stronger than the cuebid.  In common practice, the only time this is relevant is over \he1--(2NT), where \di3 showing spades but not strong (less than GF) allows for a \he3 rebid (NF). \sp3 in that instance is forcing.	

\item \textbf{2 Suiter, 1 unknown} We treat this auction as if they had bid the one known suit. Cuebid is support for opener, new suits forcing, etc.

\item \textbf{2NT} In competition, 2NT might mean many different things depending on context. When the bidder is ``forced'' (i.e., responding to a takeout style double) the default meaning is Scrambling. When 2NT is a free bid it is most commonly Good/Bad; one notable exception is when 2NT is the first bid by responder, in which case it is natural except where otherwise defined. (i.e., 1M-Dbl).

One other possibility in ``Good/Bad'' sequences where Good isn't possible (hand already limited severely): secondary but higher ranked suit. These need to be better defined. 
\end{itemize}

\section{Misc}

This section is for items which don't fit elsewhere.

\vspace{1em}\hrule

\subsection{XX of cuebids}


When does XX promise \first round control? Does it matter if last train is in play?

Auction (with Jenni)

\auction[Jenni,,Tom,]{,,1d,1h,2h,3h,3s,p,4c,p,4d,p,4h,x,p,p,xx,p,4s,p,4n,p,5s,p,6s}

First, Jenni was uncertain that \cl4 was Serious/Non-Serious. I think since we aren't in a GF that NS is off, but it is ambiguous.

Second, my pass over \he4-Dbl was clearly encouraging, i.e. last train. What does that mean, if anything, for the XX? \first? Counter last train?

Slam was good today, but hardly the point. 

\begin{handdiagram}
	\north{52,a763,832,k987}
	\east[Tom]{kqt,j84,aqt64,j3}
	\south{j3,kqt95,97,t652}
	\west[Jenni]{a98764,2,kj5,aq4}
\end{handdiagram}

Jenni was concerned about her trumps, I liked my hand but felt I had little I could do with nothing left to cuebid other than pass to encourage. She eventually decided that my pass must mean my trumps are good enough and bid RKC. I suspect her XX is flawed on this particular deal, but it does bring to question whether last train should overrule the general rule about showing \first.

\vspace{1em}\hrule

\end{document}