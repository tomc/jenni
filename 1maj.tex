\documentclass[tom-jenni]{subfile}
\begin{document}
	
\chapter{1 Major}
		
\section{Intro}
	
Few changes here from old version.  General approach is natural with semi-forcing NT and using \cl2 and \di2 as artificial bids; \cl2 is a GF minor hand, \di2 is TaJ.

\section{Response Summary (UPH)}

\begin{bidtable}{\orauction{1h/\sss}}
	\sp1 & Natural, F1 \\
	1NT & Semi-forcing, does not include limit raises.  Commonly the only invite is balanced. \\
	\cl2 & GF with either or both minors, artificial \\
	\di2 & TaJ, Limit + in Opener's major \\
	R & Simple Raise \\
	JS & Jump shifts (including \he1-\sp2) are natural and game invitational \\
	DR & Mixed \\
	DJS & Void Splinter; regular splinters start with TaJ \\
	2NT & Natural GF \\
	3NT & 17-18 Natural \\
\end{bidtable}

\section[2C Minor GF]{\cl2 Minor GF}

To make room to use \di2 as TaJ, we use \cl2 as our GF for either minor.  The balanced GFs go through 2NT, so this is only for the ``real'' minor hands.  Responses are fairly natural with the exception of \di2, which is waiting denying the other hand types.

It is expected that Responder will not have 3 card support for Opener's major.

\begin{bidtable}{\orauction{1h/\sss,2c}}
	\di2 & Waiting, typically denies the other listed hand types \\
	2M & 6+ in the Major \\
	2OM & 4 in the other Major \\
	2NT & Rarely used, but natural \\
	3x & 5-5 \\
	3M & Strong suit, setting trumps \\	
\end{bidtable}

Responder has some structured rebids as well.  Many of these only apply over the \di2 waiting bid.

\begin{bidtable}{\orauction{1h/\sss,2c,2d}}
	2M & Honor doubleton \\
	2OM & Natural, long minor still ambiguous \\
	2NT & Usually 5-4 in the minors \\
	3m & 6+ \\
	3M & Shortness, 5-5 in the minors (Only over \di2) \\
\end{bidtable}

\section{Passed Hand}

Things revert to natural by a PH.  A raise to 2M is our strongest bid, generally constructive.  With a weaker hand we just pass the 1 bid.

Jump shifts are fit.
	  
\end{document}


