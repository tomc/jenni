\documentclass[tom-jenni]{subfiles}
\begin{document}
	
\chapter[1D]{\di1}
	
\di1 is our catch-all opening bid for hands with no 5-card major and fewer than 6 clubs. The range is (9)10-15 HCP if unbalanced or 10-13 HCP if balanced. \di1 does not promise any diamonds at all; \exactshape{4405} hands are routinely opened \di1.  The following hand types are included in the \di1 opener.

\begin{itemize}
  \item 10-13 HCP balanced
  \item 12-15 HCP, 6+ \ddd
  \item (9)10-15 HCP unbalanced, no 5-card major or 6-card minor
\end{itemize}

Like most of our system, we try to invite and get out as low as possible. The structure reflects this concept. We may lose some granularity in some auctions to support this style, but such is life.

\begin{bidtable}{\orauction{1d}}
  P & 0--9. It is routine to pass with up to 9 HCP and no 4-card major. \\        
  \he1 & 4+ \hhh, F1 \\
  \sp1 & 4+ \sss, F1 \\
  1NT (UPH) & 10--13 HCP, INV. No 4 card major \\
  1NT (PH) & 8--9 HCP, No 4 card major \\
  \cl2  &  10+ HCP, 5+ \ccc, F1 \\
  \di2  &  10+ HCP, 5+ \ddd, F1 \\
  \he2 & Reverse Flannery, Non-invitational. 5+ \sss, 4+ \hhh, typically 0--9 HCP \\
  \sp2 & Reverse Flannery, INV. 5+ \sss, 4+ \hhh, about 10--13 HCP \\
  2NT & Natural, GF. No 4-card major. 14--16 HCP or 19+ \\
  \cl3 & Natural, 6+ \ccc, Mixed (7--9). No suit quality requirements.\\
  \di3 & Natural, 6+ \ddd, Mixed (7--9). No suit quality requirements. \\
  \he3/\sss & "Scambled Splinter". Shortness in bid suit, at least 5--4 either way in the minors, GF. \\
  3NT & 17--18 HCP Balanced \\
\end{bidtable}

\section[1D--1M]{\di1--1M}

\di1--1M is a standard response, showing 4+ cards in the suit bid and forcing 1 round. On very rare occasions we have been known to respond in a 3-card suit with a hand like \hhand{j,ktx,kjxx,98xxx}. This sort of response is outside expectation and if responder chooses to do so they do at their own risk. Systemically this is a pass.

After \di1--\he1 opener is expected to bid \sp1 any time they have 4 spades. Again, opener my choose to bid 1NT instead, but this is also non-systemic.

\begin{bidtable}{\orauction{1d,1h}}
  \sp1 & 4 \sss. Opener is never expected to bypass a 4-card spade suit. Judgment allowed of course, but rarely would be seen outside \exactshape{4333}. \\
  1NT & 10--13 BAL. \shape{31(45)} is common as well. \\
  \cl2 & 5+ 4+ in the minors, either could be longer.  \\
  \di2  &  6+ \ddd, 12--15 HCP \\
  \he2 & Simple raise, usually (always?) 4 \hhh. 10--13 HCP if balanced or 10--14 HCP if unbalanced. \\
  \sp2 & Natural, shapely. 5--6 or better 13--15, NF. \\
  2NT & 6+ \ddd~ 3 \hhh. Might rarely be 6--4 with the ``standard'' \di4 bid \\
  \cl3 & 5+ \ddd~ 5+ \ccc (13)14--15 HCP, NF \\
  \di3 & 6+ \ddd, good hand. Note that \di2 is already more than a minimum, so this is a very strong hand/suit. \\
  \he3 & 4 \hhh, unbalanced, typically (13)14--15 HCP \\
  \sp3 & Spl \\
  \cl4/\ddd & Spl \\
\end{bidtable}

Opener's rebids after \di1--\sp1 are similar:

\begin{bidtable}{\orauction{1d,1s}}
  1NT & 10--13 BAL. Singleton spade is common as well. \\
  \cl2 & Typically 5+ 4+ in the minors, although \exactshape{14xx} is possible with x ranging from 3 to 5.  \\
  \di2  &  6+ \ddd, 12--15 HCP \\
  \he2 & Natural, shapely. Typically 5--6 or better, 13--15 NF. \rem{Many pairs play this as a 3 card raise to avoid NT rebids offshape and be explicit. Meckwell have it overloaded as showing a few different strong hands.} \\
  \sp2 & Simple raise, usually 4 \sss. 10--13 HCP if balanced or 10--14 HCP if unbalanced. \\
  2NT & 6+ \ddd~ 3 \sss. Might rarely be 6--4 with the ``standard'' \di4 bid \\
  \cl3 & 5+ \ddd~ 5+ \ccc~ 14--15 HCP \\
  \di3 & 6+ \ddd, good hand. Note that \di2 is already more than a minimum, so this is a very strong hand/suit. \\
  \he3 & Spl \\
  \sp3 & 4 \sss, unbalanced, typically (13)14--15 HCP \\
  \cl4/ddd & Spl \\
  \he4 & \rem{?} \\
\end{bidtable}

\section[1D--1NT]{\di1--1NT}

1NT is game invitational, 10--13. This is an attribute of TaJ that is quite dissimilar from most strong club systems.

Generally speaking most auctions will end up either in 1NT or 3NT, but there options to handle other hand types.

\begin{bidtable}{\orauction{1d,1n}}
	\cl2 & To play, does not imply \ddd \\
	\di2 & To play, presumably only 5 \ddd~ (no \di2 opener) \\
	\he2 & Art GF, unbalanced. Typically 5 \ccc~ with \shape{5431} or \shape{5440}. Also includes any \shape{4441}, including singleton \ccc \\
	\sp2 & Art GF, 5+ \ddd~ unbalanced or semi-bal. If 6+ \ddd~ then no other 4 card suit. \\
	2NT & Re-invite. Typically 12--13 bal. \\
	\cl3 & 5--5 minors, GF. \\
	\di3 & 6--4 minors, GF. \\
	\he3 & 6--4 natural, GF. \\
	\sp3 & 6--4 natural, GF. \\
	3NT & To play. \\
\end{bidtable}

Over \he2, \sp2 asks. 2NT shows any 4x1 (\cl3 asks UTL) otherwise LMH shortness with 5 \ccc.

Over \sp2, 2NT asks LMHN shortness.  None is rare, so boosted to be the final step.  

\section[1D--2m]{\di1--2m}

A \cl2/\di2 response are both similar, natural and forcing 1 round, typically 10+. (Inv+)

We play a modified Meckwell structure, using artificial rebids. Other than \he2, all bids promise a non-minimum.

\begin{bidtable}{\orauction{1d,2m}}
	\he2 & Any minimum. (Different from Meckwell) Over this \sp2 is ``Lebensohl'', requesting 2NT for sign off there or in a minor. Opener may bid 3 of a minor to escape instead of 2NT. \\
	\sp2 & GF, Unspecified splinter raise of responder's minor. 2NT asks LMH. \\
	2NT & Typically 12--13 bal. 3m rebid non-forcing. \\
	om,R &  Natural, non-min. \\
\end{bidtable}

Generally speaking, rebidding the minor by responder is NF. The exception is over the \he2 minimum bid, where \sp2 starts all weak sequences and 3 of a minor directly (new or old) is forcing.

\section[1D--2H]{\di1--\he2}

5+ spades, 4+ hearts, less than invitational. 

Generally either passed or corrected to \sp2. Extreme hands might invite with 3M. No discussion about other bids, 3m is presumably non-forcing.

\section[1D--2S]{\di1--\sp2}

5+ spades, 4+ hearts, invitational.  2NT is ``Pref-Lebensohl'', responder bids the minor they prefer. Direct 3 level bids are forcing (minor) or forward going but NF (major).

\section[1D--2NT]{\di1--2NT}

GF balanced. No special methods at this time. 13+ to 16 or 19+.  \cl4 is Gerber (1430).

\section[1D--3m]{\di1--3m}

Mixed strength, 6+ cards.

\section[1D--3M]{\di1--3M}

Splinter with both minors, at least \shape{xx(54)}, GF.

\section[1D--3NT]{\di1--3NT}

17--18 balanced. No special methods. \cl4 is Gerber (1430).

\section{Other}

4M natural and to play.

4m is South African Texas:  \cl4=\hhh, \di4=\sss 

\end{document}



