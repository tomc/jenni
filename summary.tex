\documentclass[tom-jenni]{subfile}
\begin{document}
	
\chapter{System Summary}


''TaJ Precision'' 
Strong Club system with 0+ \di1.  No "Precision" \di2 opener.  Most 10 HCP hands opened NV, allowed to pass a 10 Vul. Lighter openers possible with shape.  

\section{Opening Bid Summary}

\begin{description}
	\item[\cl1] Strong, Forcing, Artificial.  Typically 16+ HCP unbal or 17+ bal
	\item[\di1] 0+ \ddd, 10-15.  Denies 5 card major unless 6+ \ddd
	\item[\he1] 5+ \hhh, 10-15
	\item[\sp1] 5+ \sss, 10-15
	\item[1NT] 14-16.  5 card major, 6 card minor, \shape{5422} common.
	\item[\cl2] 6+ \ccc, 10-15.  5 card suit possible in \third seat for lead direction.
	\item[\di2] 6+ \ddd, (8)9-12.  5 card suit possible in \third seat for lead direction.
	\item[\he2/\sss] 3-9, 5 card suit common NV, 7 card suit uncommon but possible Vul
	\item[2NT] 22-23
	\item[3x] Natural, aggressive
	\item[3NT] Good Major preempt.  (Namyats-like)
	\item[4x] Natural, aggressive
\end{description}

No special agreements for opening bids 4NT and higher.

\section{General Principles}

\begin{itemize}
	\item Doubles unless otherwise defined are takeout
	\item All Strange Bids are Forcing (ASBAF).  A general guideline to cover the unknown.
	\item In an auction where we've committed to a certain level and the opps interfere (including double), pass is generally more encouraging than bidding to the forced the level.  A common example is a cuebid (raise) being doubled, then rebidding our trump suit.
	\item We ignore most doubles by the opponent, bidding retains their meaning.  This does not apply to 1-level suit opening bids.
	\item Minimum responses to opening bids: while we pass \di1 freely up to 9 HCP, we follow more standard approaches to responding to 1M: respond with an Ace, a King and a 5 count or any 6 count.  Responding to 2m is a little different, passing is quite possible with 8 or so points, especially with no fit.  Even 10 or 11 counts are possible over \di2.
\end{itemize}	

\section{Relays}

\subsection{TaJ}
TaJ relay as it currently exists.  Used in both \cl1 auctions and 1M--[raise] auctions.

\begin{description}
	\item[Special] In auctions where responder is unlimited, first step shows extra values.  Next step repeats TaJ and mirrors the limited relay.
	\item[+1] \shape{54xx} Relay for \second suit LMH, then shape NLH.  Immediately ``zooming'' past the \second suit LMH relay shows LMH void and \shape{5440}
	\item[+2] \shape{55xx} or better.  Secondary suit is always equal or shorter.  Relay for \second suit LMH, then shortness LHB.
	\item[+3] \shape{64xx}.  Primary suit can be longer, secondary always 4.  Relay for \second suit LMH, then shortness LHB.
	\item[+4] 6+ card suit with shortness, denies 4 card side suit.  Relay for short suit LMH.
	\item[+5] \shape{5322}
	\item[+6] \shape{6322} or \shape{7222}.  This may also explode into further descriptive items, such as cuebids.  
\end{description}

\begin{noted}[Preserve all steps]
  We have decided to \textbf{never} drop the 5332 step in TaJ, even when it might make sense.  This keeps things clean relay wise and also allows for flexible decisions by responder in some instances.
\end{noted}

\end{document}




